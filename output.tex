% Options for packages loaded elsewhere
\PassOptionsToPackage{unicode}{hyperref}
\PassOptionsToPackage{hyphens}{url}
%

\documentclass[twoside,8pt]{extarticle}
\usepackage{graphicx} % Required for inserting images
\usepackage[paperheight=5.5in,paperwidth=3.5in,outer=8mm,inner=5mm, twoside, top=5mm, bottom=8mm, includefoot, footskip=5mm]{geometry} % removed showframe
\usepackage[document]{ragged2e} % ragged left-align
\usepackage{xcolor}


% -------------------------------------- TightList --------------------------------------
% used by convertion engine
\providecommand{\tightlist}{%
% \setlength{\itemsep}{0pt}\setlength{\parskip}{0pt}
}

% -------------------------------------- Fonts --------------------------------------

% define custom fonts

\usepackage{fontspec}
\usepackage[abspath]{currfile}

\defaultfontfeatures[Nova]
{
    Path          = \currfileabsdir,
    UprightFont   = fonts/nova/CopenhagenGroteskNovaRegular.otf,
    BoldFont      = fonts/nova/CopenhagenGroteskNovaBold.otf,
    ItalicFont    = fonts/nova/CopenhagenGroteskNovaRegular.otf
}

\defaultfontfeatures[LinuxLibertine]
{
    Path          = \currfileabsdir,
    UprightFont   = fonts/LinuxLibertine.ttf,
    BoldFont      = fonts/LinuxLibertineSemiBold.ttf,
    ItalicFont    = fonts/LinuxLibertineItalic.ttf
}

\defaultfontfeatures[Futura]
{
    Path          = \currfileabsdir,
    UprightFont   = fonts/futura-bold.ttf
}

\defaultfontfeatures[Capitals]
{
    Path          = \currfileabsdir,
    UprightFont   = fonts/LinuxLibertineCapitals-Bold-Italic.ttf
}

\newfontfamily\nova{Nova}
\newfontfamily\futurabold{Futura}
\newfontfamily\smallcaps{Capitals}

\newfontfamily\dejavu{DejaVu Sans}
\newfontfamily\source{Source Code Pro}



%\setmainfont{LinuxLibertine}
\setmainfont{Nova}

% custom font commands

\newcommand{\normfont}{%
    \fontsize{8pt}{8pt}\nova}

\renewcommand{\normalfont}{%
    \fontsize{8pt}{8pt}\nova}

\newcommand{\mediumish}{%
    \fontsize{7pt}{7pt}\futurabold}
\newcommand{\footish}{%
    \fontsize{5pt}{5pt}\nova}
\newcommand{\symbols}{%
    \fontsize{7pt}{8pt}\dejavu}


\newcommand{\latinish}[1]{\smallcaps\addfontfeature{LetterSpace=-5}#1\normfont}


% -------------------------------------- Titles --------------------------------------

\usepackage[pagestyles]{titlesec}

\titleformat{\section}[block]  % which section command to format
{\fontsize{11}{13}\nova\filcenter} % format for whole line
{} % how to show number - \thesubsection
{0em} % space between number and text
{\color{black}\textbf} % formatting for just the text
[\normfont] % formatting for after the text

\titleformat{\subsection} % which section command to format
{\fontsize{8}{10}\nova} % format for whole line
{} % how to show number - \thesubsection
{0em} % space between number and text
{\color{black}\textbf} % formatting for just the text
[\normfont] % formatting for after the text

% 
\titlespacing*{\section}
{0pt} % left
{1pt} %{<before-sep>}
{1pt} %{<after-sep>}

\titlespacing*{\subsection}
{0pt} % left
{0pt} %{<before-sep>}
{0pt} %{<after-sep>}





% -------------------------------------- Special chars --------------------------------------

\newcommand\lt{\symbols→ \normfont}
\newcommand\inc{\symbols↑\normfont}
\newcommand\dec{\symbols↓\normfont}
\newcommand\textdelta{\symbols ∆ \normfont}
\newcommand\therefor{\textbf{\symbols^^^^2234 \normfont}}


% -------------------------------------- Footer --------------------------------------
\usepackage{fancyhdr}
\pagestyle{fancy}
\fancyfoot{} % clear all footer fields
\fancyfoot[LE,RO]{\fontsize{7}{7} \nova \thepage}



% -------------------------------------- Itemize --------------------------------------
% https://www.overleaf.com/learn/latex/Lists

\usepackage{enumitem}
\setlist{nosep} % or \setlist{noitemsep} to leave space around whole list
% \setlist[legal]{label*=\arabic*.}
\setlist{label=-, nosep, leftmargin = *, itemindent=0pt, labelwidth=0pt, rightmargin=0pt}



% ---------- margin box

\usepackage{tikz}

\definecolor{MyBrown}{HTML}{A07F64}
\definecolor{MyBlue}{HTML}{6894B4}



\newcommand{\ColorMargin}[3]{% takes color and text
\ifodd\value{page}% 
\put (\paperwidth-5mm,#3){%
\begin{tikzpicture}[baseline=(current bounding box.north)]
    \node[rotate=270, minimum height= 5mm, minimum width=1.7 cm, fill=#1, text=white] at (0, 0) {\bfseries\mediumish #2};
\end{tikzpicture}}%
\else
\put (0 cm,#3){% relative to upper left corner
\begin{tikzpicture}[baseline=(current bounding box.north)]
    \node[rotate=90, minimum height= 5mm, minimum width=1.7 cm, fill=#1, text=white] at (0, 0) {\bfseries\mediumish #2};
\end{tikzpicture}}%
\fi
}


% -------------------------------------- Front Matter --------------------------------------
\title{OB handbook}
\author{alexgoodell }
\date{May 2023}

\begin{document}

\AddToHook{shipout/background}{\ColorMargin{MyBrown}{Intro}{-1cm}}

\section{Introduction}\label{introduction}

OB is quite a bit different that other MSD rotations you may have
completed. There is a slightly different culture within the division,
which can take some adjustment. Overall, the OB Division prioritizes the
following:

\begin{itemize}
\tightlist
\item
  \textbf{Timeliness}: arrive for shifts, MDR's (multi-disciplinary
  rounds), and educational events on time. If you are two minutes late,
  you will be late. Be efficient and do not let your to-do list grow by
  more than a few items, as the volume is quite unpredictable.
\item
  \textbf{Professionalism}: The diversity of patients here requires a
  flexible and understanding mindset, and respecting each patient will
  look different. Do your best to meet their needs and show your respect
  for them as much as you can.
\item
  \textbf{Protocols}: The division has done extensive research and/or
  literature reviews for almost every part of OB anesthesia. Given the
  homogeneity in cases, protocols have been developed for almost every
  aspect of the job; this allows additional research to be conducted
  easily. For the most part, unless you have a strong reason to stray
  from the protocol, you are expected to follow it.
\item
  \textbf{Follow-up}: Every patient who receives an anesthetic is
  expected to have a follow-up (ideally in-person) documented in their
  chart. Complications should be followed to resolution.
\item
  \textbf{Proactive management of pain}: Effective treatment of labor
  pain is a hallmark of the division, and inadequate treatment of pain
  is seen as a urgent problem requiring immediate attention.
\item
  \textbf{Triage}: Since the residents are often the first point of
  contact for nursing and OB, it is important to stay in close
  communication with your team and delegate tasks. See below.
\end{itemize}

\section{Triage and Roles}\label{triage-and-roles}

The OB anesthesia team is quite different from other services you may
have rotated on, which can cause some confusion for the first few days.
Residents hold the primary phones used to contact the team. The vast
majority of requests will come from nursing. Unlike other services, the
OB fellows are not to be seen as your supervisors and/or seniors--their
participation in day-to-day clinical care is minimal and for the most
part relegated to high-risk OB cases. The attending should be your first
call, even for tasks that do not require their supervision; they expect
to be called to offload your to-do list for any time-sensitive requests.
For example--if you are called to see two patients for consult, then get
another call for help with an IV, call or text your attending to help
with the IV, or direct the caller to them. Having more than a few
to-do's lined up is problematic when a more urgent task (ie, stat CS)
suddenly removes you from the floor.

\section{Introduction}\label{introduction-1}

OB is quite a bit different that other MSD rotations you may have
completed. There is a slightly different culture within the division,
which can take some adjustment. Overall, the OB Division prioritizes the
following:

\subsection{Preparation}\label{preparation}

\begin{itemize}
\tightlist
\item
  When you first arrive for your shift and complete signout, you should
  check the status of the Ors and ensure that the ORs are set up to your
  liking.
\item
  Whenever a case finishes, circle back to set up the appropriate ORs.
\item
  Resources can be limited, especially on the weekends and nights. It is
  best to become comfortable with the environment during the day so that
  you can respond quickly. This includes knowing where emergency drugs
  are kept (interlipid, code drugs, nitro spray), how the carts are
  organized, where OmniCells are.
\end{itemize}

\subsection{Preparation}\label{preparation-1}

\begin{itemize}
\tightlist
\item
  When you first arrive for your shift and complete signout, you should
  check the status of the Ors and ensure that the ORs are set up to your
  liking.
\item
  Whenever a case finishes, circle back to set up the appropriate ORs.
\item
  Resources can be limited, especially on the weekends and nights. It is
  best to become comfortable with the environment during the day so that
  you can respond quickly. This includes knowing where emergency drugs
  are kept (interlipid, code drugs, nitro spray), how the carts are
  organized, where OmniCells are.
\end{itemize}

\section{Triage and Roles}\label{triage-and-roles-1}

The OB anesthesia team is quite different from other services you may
have rotated on, which can cause some confusion for the first few days.
Residents hold the primary phones used to contact the team. The vast
majority of requests will come from nursing. Unlike other services, the
OB fellows are not to be seen as your supervisors and/or seniors--their
participation in day-to-day clinical care is minimal and for the most
part relegated to high-risk OB cases. The attending should be your first
call, even for tasks that do not require their supervision; they expect
to be called to offload your to-do list for any time-sensitive requests.
For example--if you are called to see two patients for consult, then get
another call for help with an IV, call or text your attending to help
with the IV, or direct the caller to them. Having more than a few
to-do's lined up is problematic when a more urgent task (ie, stat CS)
suddenly removes you from the floor.

\section{Clinical Responsibilities}\label{clinical-responsibilities}

\subsection{Preparation}\label{preparation-2}

\begin{itemize}
\tightlist
\item
  When you first arrive for your shift and complete signout, you should
  check the status of the Ors and ensure that the ORs are set up to your
  liking.
\item
  Whenever a case finishes, circle back to set up the appropriate ORs.
\item
  Resources can be limited, especially on the weekends and nights. It is
  best to become comfortable with the environment during the day so that
  you can respond quickly. This includes knowing where emergency drugs
  are kept (interlipid, code drugs, nitro spray), how the carts are
  organized, where OmniCells are.
\end{itemize}

\section{Epidurals/Labor Analgesia}\label{epiduralslabor-analgesia}

\begin{itemize}
\tightlist
\item
  Encounter: You will be called by the nurse to either perform a
  ``consult'' or place an epidural. A consult is a blank anesthetic
  record that is made to generate a preop note; these are done to
  document history/physical for patients even if they are not interested
  in an epidural. If they require anesthesia, you will be their
  anesthesiologists so it is important to identify any major problems
  beforehand. If the patient wants an epidural immediately, the consult
  encounter is not needed and your can just open a ``labor analgesia''
  encounter, and use the preop note within that encounter.
\item
  Based on their history, decide whether a plain epidural,
  dural-puncture epidural, or combined spinal-epidural would be best for
  them.
\item
  Discuss the procedure, as well as the risks and benefits with the
  patient and any family members present.
\item
  Obtain your medication and epidural cart, set up the kit and call your
  attending. If you are sterile, the nurses are happy to call your
  attending for you.
\item
  Target pain score is three or less (five or less during pushing), and
  tolerable for the patient.
\item
  After placing an epidural or CSE, do not leave until the patient has
  started to feel some relief. If their pain score is above 3 during
  labor or is intolerable, continue to check in and treat as needed.
  Round on patients frequently (q2-3 hours) and document in Epic as
  ``epidural check.''
\item
  Apart from bedside rounding, pain scores are documented in epic in the
  ``Pain summary'' as well as a ``sparkline report'' on the OB home
  page.
\item
  You need to complete a minimum of 5 blocks under ultrasound-guidance
  and 5 blocks in the lateral position (second rotation only).
\item
  Please keep track of your lateral and ultrasound blocks on the Google
  Sheet Document. Find your tab at the bottom and track your blocks for
  the month. Your evaluations will be submitted after Gill Abir sees you
  have completed your sheet. In the logbook please be sensitive to HIPPA
  when recording patient details - i.e.~just name initials will be fine.
\end{itemize}

\subsection{Other clinical tasks}\label{other-clinical-tasks}

\begin{itemize}
\tightlist
\item
  \textbf{Quality Improvement}: Be diligent about recording QI events in
  Epic and modifying QI notes if issues arise later. This will need to
  be done with every procedure and notable event. This is an essential
  part of our practice and your learning.
\item
  Follow up: Do post-op checks and fill in the Epic `Follow-Up' note. If
  the post-op checks are delayed due to a busy L+D floor, call the
  fellows to advise and make a team plan for getting these done.
\item
  Charting: Ensure your Epic charting is kept up-to-date, ready and
  available for your Attending to sign and attest. Be especially careful
  to enter ``Anesthesia Stop Times'' after deliveries (preferably 15
  minutes afterwards), and ensure that all essential components of a
  procedure or follow-up note are completed.
\end{itemize}

\RemoveFromHook{shipout/background}
\AddToHook{shipout/background}{\ColorMargin{MyBlue}{Physiology}{-2.8cm}}

\section{Epidurals/Labor Analgesia}\label{epiduralslabor-analgesia-1}

\begin{itemize}
\tightlist
\item
  Encounter: You will be called by the nurse to either perform a
  ``consult'' or place an epidural. A consult is a blank anesthetic
  record that is made to generate a preop note; these are done to
  document history/physical for patients even if they are not interested
  in an epidural. If they require anesthesia, you will be their
  anesthesiologists so it is important to identify any major problems
  beforehand. If the patient wants an epidural immediately, the consult
  encounter is not needed and your can just open a ``labor analgesia''
  encounter, and use the preop note within that encounter.
\item
  Based on their history, decide whether a plain epidural,
  dural-puncture epidural, or combined spinal-epidural would be best for
  them.
\item
  Discuss the procedure, as well as the risks and benefits with the
  patient and any family members present.
\item
  Obtain your medication and epidural cart, set up the kit and call your
  attending. If you are sterile, the nurses are happy to call your
  attending for you.
\item
  Target pain score is three or less (five or less during pushing), and
  tolerable for the patient.
\item
  After placing an epidural or CSE, do not leave until the patient has
  started to feel some relief. If their pain score is above 3 during
  labor or is intolerable, continue to check in and treat as needed.
  Round on patients frequently (q2-3 hours) and document in Epic as
  ``epidural check.''
\item
  Apart from bedside rounding, pain scores are documented in epic in the
  ``Pain summary'' as well as a ``sparkline report'' on the OB home
  page.
\item
  You need to complete a minimum of 5 blocks under ultrasound-guidance
  and 5 blocks in the lateral position (second rotation only).
\item
  Please keep track of your lateral and ultrasound blocks on the Google
  Sheet Document. Find your tab at the bottom and track your blocks for
  the month. Your evaluations will be submitted after Gill Abir sees you
  have completed your sheet. In the logbook please be sensitive to HIPPA
  when recording patient details - i.e.~just name initials will be fine.
\end{itemize}

\subsection{Other clinical tasks}\label{other-clinical-tasks-1}

\begin{itemize}
\tightlist
\item
  Quality Improvement: Be diligent about recording QI events in Epic and
  modifying QI notes if issues arise later. This will need to be done
  with every procedure and notable event. This is an essential part of
  our practice and your learning.
\item
  Follow up: Do post-op checks and fill in the Epic `Follow-Up' note. If
  the post-op checks are delayed due to a busy L+D floor, call the
  fellows to advise and make a team plan for getting these done.
\item
  Charting: Ensure your Epic charting is kept up-to-date, ready and
  available for your Attending to sign and attest. Be especially careful
  to enter ``Anesthesia Stop Times'' after deliveries (preferably 15
  minutes afterwards), and ensure that all essential components of a
  procedure or follow-up note are completed.
\end{itemize}

\section{Epidurals/Labor Analgesia}\label{epiduralslabor-analgesia-2}

\begin{itemize}
\tightlist
\item
  Encounter: You will be called by the nurse to either perform a
  ``consult'' or place an epidural. A consult is a blank anesthetic
  record that is made to generate a preop note; these are done to
  document history/physical for patients even if they are not interested
  in an epidural. If they require anesthesia, you will be their
  anesthesiologists so it is important to identify any major problems
  beforehand. If the patient wants an epidural immediately, the consult
  encounter is not needed and your can just open a ``labor analgesia''
  encounter, and use the preop note within that encounter.
\item
  Based on their history, decide whether a plain epidural,
  dural-puncture epidural, or combined spinal-epidural would be best for
  them.
\item
  Discuss the procedure, as well as the risks and benefits with the
  patient and any family members present.
\item
  Obtain your medication and epidural cart, set up the kit and call your
  attending. If you are sterile, the nurses are happy to call your
  attending for you.
\item
  Target pain score is three or less (five or less during pushing), and
  tolerable for the patient.
\item
  After placing an epidural or CSE, do not leave until the patient has
  started to feel some relief. If their pain score is above 3 during
  labor or is intolerable, continue to check in and treat as needed.
  Round on patients frequently (q2-3 hours) and document in Epic as
  ``epidural check.''
\item
  Apart from bedside rounding, pain scores are documented in epic in the
  ``Pain summary'' as well as a ``sparkline report'' on the OB home
  page.
\item
  You need to complete a minimum of 5 blocks under ultrasound-guidance
  and 5 blocks in the lateral position (second rotation only).
\item
  Please keep track of your lateral and ultrasound blocks on the Google
  Sheet Document. Find your tab at the bottom and track your blocks for
  the month. Your evaluations will be submitted after Gill Abir sees you
  have completed your sheet. In the logbook please be sensitive to HIPPA
  when recording patient details - i.e.~just name initials will be fine.
\end{itemize}

\subsection{Other clinical tasks}\label{other-clinical-tasks-2}

\begin{itemize}
\tightlist
\item
  Quality Improvement: Be diligent about recording QI events in Epic and
  modifying QI notes if issues arise later. This will need to be done
  with every procedure and notable event. This is an essential part of
  our practice and your learning.
\item
  Follow up: Do post-op checks and fill in the Epic `Follow-Up' note. If
  the post-op checks are delayed due to a busy L+D floor, call the
  fellows to advise and make a team plan for getting these done.
\item
  Charting: Ensure your Epic charting is kept up-to-date, ready and
  available for your Attending to sign and attest. Be especially careful
  to enter ``Anesthesia Stop Times'' after deliveries (preferably 15
  minutes afterwards), and ensure that all essential components of a
  procedure or follow-up note are completed.
\end{itemize}

\section{Epidurals/Labor Analgesia}\label{epiduralslabor-analgesia-3}

\begin{itemize}
\tightlist
\item
  Encounter: You will be called by the nurse to either perform a
  ``consult'' or place an epidural. A consult is a blank anesthetic
  record that is made to generate a preop note; these are done to
  document history/physical for patients even if they are not interested
  in an epidural. If they require anesthesia, you will be their
  anesthesiologists so it is important to identify any major problems
  beforehand. If the patient wants an epidural immediately, the consult
  encounter is not needed and your can just open a ``labor analgesia''
  encounter, and use the preop note within that encounter.
\item
  Based on their history, decide whether a plain epidural,
  dural-puncture epidural, or combined spinal-epidural would be best for
  them.
\item
  Discuss the procedure, as well as the risks and benefits with the
  patient and any family members present.
\item
  Obtain your medication and epidural cart, set up the kit and call your
  attending. If you are sterile, the nurses are happy to call your
  attending for you.
\item
  Target pain score is three or less (five or less during pushing), and
  tolerable for the patient.
\item
  After placing an epidural or CSE, do not leave until the patient has
  started to feel some relief. If their pain score is above 3 during
  labor or is intolerable, continue to check in and treat as needed.
  Round on patients frequently (q2-3 hours) and document in Epic as
  ``epidural check.''
\item
  Apart from bedside rounding, pain scores are documented in epic in the
  ``Pain summary'' as well as a ``sparkline report'' on the OB home
  page.
\item
  You need to complete a minimum of 5 blocks under ultrasound-guidance
  and 5 blocks in the lateral position (second rotation only).
\item
  Please keep track of your lateral and ultrasound blocks on the Google
  Sheet Document. Find your tab at the bottom and track your blocks for
  the month. Your evaluations will be submitted after Gill Abir sees you
  have completed your sheet. In the logbook please be sensitive to HIPPA
  when recording patient details - i.e.~just name initials will be fine.
\end{itemize}

\subsection{Other clinical tasks}\label{other-clinical-tasks-3}

\begin{itemize}
\tightlist
\item
  Quality Improvement: Be diligent about recording QI events in Epic and
  modifying QI notes if issues arise later. This will need to be done
  with every procedure and notable event. This is an essential part of
  our practice and your learning.
\item
  Follow up: Do post-op checks and fill in the Epic `Follow-Up' note. If
  the post-op checks are delayed due to a busy L+D floor, call the
  fellows to advise and make a team plan for getting these done.
\item
  Charting: Ensure your Epic charting is kept up-to-date, ready and
  available for your Attending to sign and attest. Be especially careful
  to enter ``Anesthesia Stop Times'' after deliveries (preferably 15
  minutes afterwards), and ensure that all essential components of a
  procedure or follow-up note are completed.
\end{itemize}

\RemoveFromHook{shipout/background}
\end{document}